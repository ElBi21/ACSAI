\chapter{MATLAB}

MATLAB è un linguaggio di programmazione sviluppato negli anni '70, che viene usato per sviluppare modelli matematici, svolgere simulazioni e analisi dei dati. Il modo in cui MATLAB funziona è detto \textbf{interattivo}, poiché viene tutto eseguito nella console. È possibile anche eseguire più comandi assieme nello stesso prompt, separando tutti i comandi con delle virgole. Ad esempio:

\begin{terminal}
    \begin{lstlisting}[language=MATLAB, style=notexterm]
>> 2+3, 7*2, 9+1*3\end{lstlisting}
    \tcblower
    \begin{lstlisting}[style=notexterm]
ans = 5
ans = 14
ans = 12\end{lstlisting}
\end{terminal}

Operazioni più lunghe possono essere scritte su più righe usando i "\texttt{...}". Gli operatori disponibili sono i seguenti:

\begin{center}
    \begin{tabular}{|c|c|}
        \hline
        \textbf{Operazione} & \textbf{Operatore} \\
        \hline \hline
        Somma & \texttt{+} \\
        \hline
        Sottrazione & \texttt{-} \\
        \hline
        Moltiplicazione & \texttt{*} \\
        \hline
        Divisione & \texttt{/} \\
        \hline
        Potenza & \verb|^| \\
        \hline\hline
        Minore & \texttt{<} \\
        \hline
        Maggiore & \texttt{>} \\
        \hline
        Minore o uguale & \texttt{<=} \\
        \hline
        Maggiore o uguale & \texttt{>=} \\
        \hline
        Uguale & \texttt{==} \\
        \hline
        Diverso & \verb|~=| \\
        \hline
    \end{tabular}
\end{center}

In MATLAB, è possibile anche usare gli operatori logici, quali l'\texttt{AND}, l'\texttt{OR} e il \texttt{NOT}. Chiaramente, anche i gate logici più complessi, che vengono costruiti con gli operatori logici più semplici, sono disponibili.

\begin{center}
    \begin{tabular}{|c|c|}
        \hline
        \textbf{Operazione} & \textbf{Operatore} \\
        \hline \hline
        \texttt{AND} & \verb|&| \\
        \hline
        \texttt{OR} & \texttt{|} \\
        \hline
        \texttt{NOT} & \verb|~| \\
        \hline
    \end{tabular}
\end{center}

Ci sono anche alcune costanti, che vengono incluse in MATLAB di default dalla libreria standard. Qui alcune di queste vengono elencate:

\begin{center}
    \begin{tabular}{|c|c|}
        \hline
        \textbf{Operazione} & \textbf{Operatore} \\
        \hline\hline
        Infinito ($\infty$) & \texttt{inf} \\
        \hline
        $\pi$ & \texttt{pi} \\
        \hline
        $i$ & \texttt{i} \\
        \hline
        Numero massimo rappresentabile & \texttt{realmax} \\
        \hline
        Numero minimo rappresentabile & \texttt{realmin} \\
        \hline
        Precisione della macchina & \texttt{eps} \\
        \hline
        Forma indeterminata / \texttt{Not A Number} & \texttt{nan} \\
        \hline
    \end{tabular}
\end{center}