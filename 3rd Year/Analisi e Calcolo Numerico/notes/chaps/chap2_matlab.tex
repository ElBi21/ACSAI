\chapter{MATLAB}

MATLAB è un linguaggio di programmazione sviluppato negli anni '70, che viene usato per sviluppare modelli matematici, svolgere simulazioni e analisi dei dati. Il modo in cui MATLAB funziona è detto \textbf{interattivo}, poiché viene tutto eseguito nella console. È possibile anche eseguire più comandi assieme nello stesso prompt, separando tutti i comandi con delle virgole. Ad esempio:

\begin{terminal}
    \begin{lstlisting}[language=MATLAB, style=notexterm]
>> 2+3, 7*2, 9+1*3\end{lstlisting}
    \tcblower
    \begin{lstlisting}[style=notexterm]
ans = 5
ans = 14
ans = 12\end{lstlisting}
\end{terminal}

Operazioni più lunghe possono essere scritte su più righe usando i "\texttt{...}". Gli operatori disponibili sono i seguenti:

\begin{center}
    \begin{tabular}{|c|c|}
        \hline
        \textbf{Operazione} & \textbf{Operatore} \\
        \hline \hline
        Somma & \texttt{+} \\
        \hline
        Sottrazione & \texttt{-} \\
        \hline
        Moltiplicazione & \texttt{*} \\
        \hline
        Divisione & \texttt{/} \\
        \hline
        Potenza & \verb|^| \\
        \hline\hline
        Minore & \texttt{<} \\
        \hline
        Maggiore & \texttt{>} \\
        \hline
        Minore o uguale & \texttt{<=} \\
        \hline
        Maggiore o uguale & \texttt{>=} \\
        \hline
        Uguale & \texttt{==} \\
        \hline
        Diverso & \verb|~=| \\
        \hline
    \end{tabular}
\end{center}

In MATLAB, è possibile anche usare gli operatori logici, quali l'\texttt{AND}, l'\texttt{OR} e il \texttt{NOT}. Chiaramente, anche i gate logici più complessi, che vengono costruiti con gli operatori logici più semplici, sono disponibili.

\begin{center}
    \begin{tabular}{|c|c|}
        \hline
        \textbf{Operazione} & \textbf{Operatore} \\
        \hline \hline
        \texttt{AND} & \verb|&| \\
        \hline
        \texttt{OR} & \texttt{|} \\
        \hline
        \texttt{NOT} & \verb|~| \\
        \hline
    \end{tabular}
\end{center}

Ci sono anche alcune costanti, che vengono incluse in MATLAB di default dalla libreria standard. Qui alcune di queste vengono elencate:

\begin{center}
    \begin{tabular}{|c|c|}
        \hline
        \textbf{Operazione} & \textbf{Operatore} \\
        \hline\hline
        Infinito ($\infty$) & \texttt{inf} \\
        \hline
        $\pi$ & \texttt{pi} \\
        \hline
        $i$ & \texttt{i} \\
        \hline
        Numero massimo rappresentabile & \texttt{realmax} \\
        \hline
        Numero minimo rappresentabile & \texttt{realmin} \\
        \hline
        Precisione della macchina & \texttt{eps} \\
        \hline
        Forma indeterminata / \texttt{Not A Number} & \texttt{nan} \\
        \hline
    \end{tabular}
\end{center}

Nella scorsa tabella, i valori \texttt{realmax} e \texttt{realmin} si riferiscono rispettivamente al valore massimo e minimo rappresentabile considerando numeri in IEEE 754 a doppia precisione (dunque a 64 bits). La libreria standard di MATLAB possiede varie funzioni matematiche, tra cui $\sin (x)$, $\cos (x)$, $\tan (x)$, $\log (x)$, etc... Nel caso in cui si volessero avere più informazioni circa una funzione, si può usare la funzione \texttt{help <funzione>}, dove \texttt{<funzione>} è la funzione di cui vogliamo ottenere più informazioni. Ad esempio:

\begin{terminal}
    \begin{lstlisting}[language=MATLAB, style=notexterm]
>> help log\end{lstlisting}
    \tcblower
    \begin{lstlisting}[style=notexterm, language = tex]
 log - Natural logarithm
    This MATLAB function returns the natural logarithm ln(x) of each element
    in array X.

    Syntax
        Y = log(X)

    Input Arguments
        X - Input array
        scalar | vector | matrix | multidimensional array | table |
        timetable

    Output Arguments
        Y - Logarithm values
        scalar | vector | matrix | multidimensional array | table |
        timetable

    Examples
        Natural Logarithm of Negative Number

    See also log1p, log2, log10, exp, logm, reallog, loglog, semilogx,
        semilogy

    Introduced in MATLAB before R2006a
    Documentation for log
    Other uses of log\end{lstlisting}
\end{terminal}

Se dovessimo aver bisogno di una funzione che svolga un certo compito, ma non ci dovessimo ricordare qual'è la funzione adatta, possiamo usare invece la funzione \texttt{lookfor <keywords>}, dove \texttt{<keywords>} è un insieme di keywords per identificare la funzione che cerchiamo. Ad esempio:

\begin{terminal}
    \begin{lstlisting}[language=MATLAB, style=notexterm]
>> lookfor square\end{lstlisting}
    \tcblower
    \begin{lstlisting}[style=notexterm, language = tex]
cgs                            - Solve system of linear equations - conjugate
                                 gradients squared method
deconv                         - Least-squares deconvolution and polynomial 
                                 division
hypot                          - Square root of sum of squares (hypotenuse)
lscov                          - Least-squares solution in presence of known 
                                 covariance
lsqminnorm                     - Minimum norm least-squares solution to 
                                 linear equation
lsqnonneg                      - Solve nonnegative linear least-squares 
                                 problem
lsqr                           - Solve system of linear equations - 
                                 least-squares method
[...]\end{lstlisting}
\end{terminal}

\section{Variabili, handling della memoria e formati}

Le variabili su MATLAB vengono assegnate e dichiarate similmente a Python: l'assegnazione e la dichiarazione avvengono allo stesso momento. Ad esempio, se volessimo dichiarare la variabile $a=4$ ci basterebbe eseguire il seguente codice:

\begin{terminal}
    \begin{lstlisting}[language=MATLAB, style=notexterm]
>> a = 4\end{lstlisting}
    \tcblower
    \begin{lstlisting}[style=notexterm, language = tex]
a = 4\end{lstlisting}
\end{terminal}

Possiamo visualizzare il contenuto di una variabile in due modi: o chiamando la variabile nella console, o usando la funzione \texttt{disp(<variabile>)}. Segue un esempio:

\begin{terminal}
    \begin{lstlisting}[language=MATLAB, style=notexterm]
>> a
>> disp(a)\end{lstlisting}
    \tcblower
    \begin{lstlisting}[style=notexterm, language = tex]
a = 4

4\end{lstlisting}
\end{terminal}

Per cancellare tutte le variabili dalla memoria si usa il comando \texttt{clear}. Per salvarne alcune tra una sessione e l'altra, possiamo usare il comando \texttt{save <filename> [<var1> <var2> ...]}, dove \texttt{<filename>} è il file in cui salveremo le variabili (in estensione \texttt{.mat}), mentre \texttt{[<var1> <var2> ...]} è una lista di variabili che vogliamo salvare. Ad esempio:

\begin{terminal}
    \begin{lstlisting}[language=MATLAB, style=notexterm]
>> b = 5;
>> c = 7;
>> save Vars/someVars b c\end{lstlisting}
\end{terminal}

Una volta cancellate le variabili dalla memoria, usando il comando \texttt{who} non le vedremmo più. Possiamo caricare nuovamente le variabili all'interno di MATLAB usando il comando \texttt{load <filename>}, che caricherà tutte le variabili all'interno del file \texttt{<filename>}. Non è necessario includere l'estensione \texttt{.mat}.

\begin{terminal}
    \begin{lstlisting}[language=MATLAB, style=notexterm]
>> load Vars/someVars
>> who\end{lstlisting}
    \tcblower
    \begin{lstlisting}[style=notexterm, language = tex]
Your variables are:

b  c  \end{lstlisting}
\end{terminal}

MATLAB ha vari formati per i dati, simili concettualmente ai tipi di Python, molto vicini ai tipi di C. Ad esempio:

\begin{center}
    \begin{tabular}{|c|c|}
        \hline
        \textbf{Formato} & \textbf{Descrizione} \\
        \hline\hline
        \texttt{double} & Numeri in doppia precisione \\
        \hline
        \texttt{uint8} & Interi senza segno a 8 bits \\
        \hline
        \texttt{uint16} & Interi senza segno a 16 bits \\
        \hline
        \texttt{uint32} & Interi senza segno a 32 bits \\
        \hline
        \texttt{int8} & Interi con segno a 8 bits \\
        \hline
        \texttt{int16} & Interi con segno a 16 bits \\
        \hline
        \texttt{int32} & Interi con segno a 32 bits \\
        \hline
        \texttt{single} & Numeri a singola precisione \\
        \hline
        \texttt{char} & Caratteri, 2 bytes per carattere \\
        \hline
        \texttt{logical} & Valore che è o \texttt{0} o \texttt{1}, generalmente usato come valore Booleano \\
        \hline
    \end{tabular}
\end{center}

Possiamo impostare anche un formato di visualizzazione dei dati tramite la funzione \texttt{format}. Tuttavia, questo formato sarà valido \textbf{solo per la visualizzazione dei dati}: all'interno di MATLAB i calcoli verranno effettuati con la stessa precisione standard di MATLAB. Facciamo un esempio:

\begin{terminal}
    \begin{lstlisting}[language=MATLAB, style=notexterm]
>> % Qui MATLAB mostrerà i dati in formato short
   sqrt(2)

>> % Con long aumentiamo il formato
   format long
   sqrt(2)\end{lstlisting}
    \tcblower
    \begin{lstlisting}[style=notexterm, language = tex]
ans = 1.4142

ans = 
   1.414213562373095\end{lstlisting}
\end{terminal}

Se volessimo cambiare il tipo dei dati, possiamo farlo usando i costruttori dei vari tipi. Ad esempio:

\begin{terminal}
    \begin{lstlisting}[language=MATLAB, style=notexterm]
>> a = 43.97;

>> int16(a)
>> double(a)
>> uint8(a)\end{lstlisting}
    \tcblower
    \begin{lstlisting}[style=notexterm, language = tex]
ans = int1644

ans = 43.9700

ans = uint844\end{lstlisting}
\end{terminal}

\section{Tipi di dati}

Oltre ai formati di numeri menzionati fino ad ora, MATLAB ha anche altri tipi di dati, che ritornano comodi per esprimere costrutti matematici come vettori, matrici e tabelle.

\subsection{Vettori, matrici e tensori}

In MATLAB è possibile usare vettori, matrici e in generale tensori a $n$ dimensioni. Come in C e in Python, vettori e matrici sono realizzabili tramite array a rispettivamente una e due dimensioni. Anche le variabili in realtà sono considerate internamente come arrays: infatti uno scalare è rappresentato tramite matrici a dimensione $1\times 1$. Possiamo realizzare vettori riga e vettori colonna, in base al carattere usato per separare i valori:
\begin{itemize}
    \item usando la virgola \texttt{,} (o degli spazi) possiamo creare vettori riga;
    \item usando il punto e virgola \texttt{;} possiamo creare vettori colonna.
\end{itemize}

\begin{terminal}
    \begin{lstlisting}[language=MATLAB, style=notexterm]
>> A = [10, 20, 30, 40, 50]
>> B = [10; 20; 30; 40; 50]\end{lstlisting}
    \tcblower
    \begin{lstlisting}[style=notexterm, language = tex]
A = 1x5
    10    20    30    40    50

B = 5x1
    10
    20
    30
    40
    50\end{lstlisting}
\end{terminal}