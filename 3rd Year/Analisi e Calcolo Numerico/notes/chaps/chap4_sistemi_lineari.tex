\chapter{Sistemi di equazioni lineari}

Nel precedente capitolo abbiamo trattato vari metodi per risolvere equazioni e sistemi di equazioni non lineari, tramite metodi come quelli di Newton-Raphson e del punto fisso. In questo capitolo ci concentreremo sulla risoluzione di sistemi di equazioni lineari, e scopriremo che in molti casi i metodi che potremo utilizzare sono simili a quelli utilizzati in precedenza. Procediamo illustrando quindi cos'è un sistema lineare:

\begin{definition}{Sistema lineare}
    Un \textbf{sistema lineare} ha la forma $AX \eq B$, dove:
    \begin{itemize}
        \item $A$ è la matrice dei coefficienti;
        \item $B$ è il vettore delle incognite;
        \item $C$ è il vettore dei termini noti.
    \end{itemize}
\end{definition}

Ci sono due tipi di metodi che possiamo utilizzare per risolvere problemi contenenti sistemi di equazioni lineari: i \textbf{metodi diretti} e i \textbf{metodi iterativi}. Inizieremo illustrando i metodi diretti, passando poi ai metodi iterativi.

\section{Metodi diretti}

I metodi diretti sono basati sulla trasformazione del sistema di partenza in un sistema equivalente, avente una struttura pre-determinata (generalmente semplice). Tali metodi garantiscono che la soluzione numerica venga calcolata \textbf{esattamente} (in assenza di errori di troncamento) in un \textbf{numero finito di passi}, tuttavia hanno un elevato consumo di RAM, e sono dunque sconsigliati nel caso di matrici con "tanti" coefficienti.
\nwl
Prima di iniziare a illustrare i vari metodi di soluzione, andremo ad approfondire il concetto di \textbf{costo computazionale}. 

\begin{definition}{Costo computazionale}
    Il \textbf{costo computazionale} $C_c$ di un algoritmo è il numero di \textbf{operazioni pesanti} (come moltiplicazioni o divisioni) che sono necessarie per calcolare la soluzione numerica.
\end{definition}

In informatica, il costo computazionale è generalmente calcolato seguendo la notazione $O(\cdot)$, che indica il numero massimo di operazioni che potrebbero esser fatte da un algoritmo nel caso peggiore (quindi si calcola il limite superiore del numero di operazioni che verrebbero eseguite).

\subsection{Metodo di Cramer}

Un metodo che è molto utilizzato nel campo dell'algebra è quello di \textbf{Cramer}:

\begin{definition}{Metodo di Cramer}
    Dato un sistema lineare in forma $AX \eq B$, se $A$ è una matrice regolare, allora attraverso l'inversa $A^{-1}$ si ha che:
    \[ X \eq A^{-1} B \quad \Longrightarrow \quad x_i \eq \frac{D_i}{\text{det}(A)} \quad \text{per } i \eq 1, \; 2, \; ..., \; n \]

    dove $D_i$ è il determinante della matrice che si ottiene da $A$ sostituendo la $i$-esima colonna col vettore $B$
\end{definition}

% inserire esempio

Il problema principale con l'algoritmo di Cramer riguarda il suo costo computazionale: per il metodo infatti, dovremo calcolare:
\begin{itemize}
    \item $n + 1$ determinanti (dunque un determinante per ogni $D_i$ e il determinante di $A$);
    \item $n!$ prodotti per ciascun determinante;
    \item $n - 1$ moltiplicazioni tra due scalari per ogni prodotto;
    \item $n$ divisioni, che diventano però trascurabili rispetto alle altre operazioni.
\end{itemize}

Il costo computazionale $C_c$ ammonta dunque a:
\[ C_c \eq \underbrace{(n + 1) \cdot (n!) \cdot (n - 1)}_{\text{moltiplicazioni}} + n \simeq (n + 1) \cdot (n!) \cdot (n - 1) \]

Per farci un'idea di quanto il numero sia in realtà alto, basta considerare un valore di $n$ come 15 o 20: con $n \eq 15$, avremmo che $C_c \simeq 3 \cdot 10^{14}$ moltiplicazioni, mentre con $n \eq 20$ avremmo $C_c \simeq 3 \cdot 10^{21}$ moltiplicazioni. Assumendo un tempo di $10^{-9}$ secondi ad operazione, avremmo che con $n \eq 15$ servirebbero 3 giorni per eseguire il metodo; con $n \eq 20$ invece il tempo richiesto per eseguire il metodo peggiora drasticamente, siccome verrebbero richiesti $3 \cdot 10^5$ anni! Questo rende il metodo praticamente inutilizzabile (poiché in quasi nessun caso $n$ sarà minore di 15).

\subsection{Metodo di eliminazione di Gauss}

Un altro metodo molto usato per risolvere i sistemi lineari è il \textbf{metodo di eliminazione di Gauss}, che cerca di trasformare il sistema in un sistema equivalente triangolare superiore:

\begin{definition}{Metodo di eliminazione di Gauss}
    Il \textbf{metodo di eliminazione di Gauss} permette di trasformare in $n - 1$ passi un sistema lineare in forma
    \[ A X \eq B \quad \text{con } X, \; B \in \mathbb{R}^n, \quad A \in \mathbb{R}^{n \times n} \]

    dove $A$ è una matrice dei coefficienti \textbf{piena}, in un sistema equivalente
    \[ U X \eq \widetilde{B} \quad \text{con } X, \; \widetilde{B} \in \mathbb{R}^n, \quad U \in \mathbb{R}^{n \times n} \]
\end{definition}

Per poter essere effettuato, il metodo di eliminazione di Gauss utilizza due operazioni: lo \textbf{scambio} di due equazioni all'interno del sistema e la \textbf{somma} di due equazioni, dove una delle due può essere moltiplicata per una costante.
\nwl
Per poter trasformare un sistema $AX \eq B$ in un sistema triangolare superiore, si usa il seguente algoritmo:

\[ UX \eq B \quad \Longrightarrow \quad \begin{cases}
    x_n \eq \frac{b_n}{u_{nn}} & \\
    x_k \eq \left( b_k - \sum_{i \eq k + 1}^n u_{ki} \cdot x_i \right) \cdot \frac{1}{u_{kk}} & k \eq n - 1, \; n - 2, \; ..., \; 2, \; 1
\end{cases} \]