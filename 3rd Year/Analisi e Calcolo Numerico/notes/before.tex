\vspace{12pt}
\section*{About these notes}

Those notes were made during my three years of university at Sapienza, and \textbf{do not} replace any professor, they can be an help though when having to remember some particular details. If you are considering of using \textit{only} these notes to study, then \textbf{don't do it}. Buy a book, borrow one from a library, whatever you prefer: these notes won't be enough.

\vspace{32pt}
\section*{License}

The decision of licensing this work was taken since these notes come from \textbf{university classes}, which are protected, in turn, by the \textbf{Italian Copyright Law} and the \textbf{University's Policy} (thus Sapienza Policy). By licensing these works I'm \textbf{not claiming as mine} the materials that are used, but rather the creative input and the work of assembling everything into one file.
\nl
All the materials used will be listed here below, as well as the names of the professors (and their contact emails) that held the courses.
\nl
The notes are freely readable and can be shared, but \textbf{can't be modified}. If you find an error, then feel free to contact me via the socials listed in my \href{https://www.leonardobiason.com}{website}. If you want to share them, remember to \textbf{credit me} and remember to \textbf{not} obscure the \textbf{footer} of these notes.

\vspace{32pt}
\section*{Bibliography \texorpdfstring{\&}{&} References}

\begingroup
    \patchcmd{\thebibliography}{\chapter*}{\section*}{}{}
    \renewcommand{\section}[2]{}%
    
    \begin{thebibliography}{9}
        \bibitem{acn1} S. C. Chapra, R. P. Canale. (2015) \emph{Numerical Methods for engineers (Seventh edition)}, McGraw Hill 
    \end{thebibliography}
\endgroup

\begin{tcolorbox}[colback=Dandelion!25, colframe=Dandelion!50]
    \begin{center}
        The "\textit{Analisi e Calcolo Numerico}" course was taught in the Spring semester in 2025 by prof. Domenico Vitulano (\href{mailto:domenico.vitulano@uniroma1.it}{\texttt{domenico.vitulano@uniroma1.it}})
    \end{center}
\end{tcolorbox}

I hope that this introductory chapter was helpful. Please reach out to me if you ever feel like. You can find my contacts on my \href{https://www.leonardobiason.com}{website}. Good luck! \nl
Leonardo Biason\\
{\footnotesize \href{mailto:leonardo@biason.org}{$\to$ \texttt{leonardo@biason.org}}}