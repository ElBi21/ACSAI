\chapter{Architetture Multicore \texorpdfstring{\&}{&} OpenMP}

Per spiegare un'architettura multicore ci vorrebbero parecchie pagine, per cui in questo capitolo ci concentreremo a vedere i punti salienti delle architetture multicore.

\section{Organizzazione delle cache}

Spesso si parla di cache quando si parla di processori, ma cosa intendiamo di preciso?

\begin{definition}{Cache}
    Una \textbf{cache} è una memoria, posizionata vicino ai core della CPU, che permette di accedere a dei dati a velocità molto più alte rispetto alla DRAM.
\end{definition}

Le cache sono fatte con una tecnologia chiamata SRAM, che è molto più rapida della DRAM (la DRAM necessita di essere caricata e scaricata continuamente, risultando in tempi di richiesta molto più alti; la SRAM invece è composta da soli transistors, motivo della velocità della cache stessa). Le memorie in SRAM sono, tuttavia, molto costose, quindi di solito le cache in SRAM sono abbastanza piccole, se comparate alla memoria principale.
\nl
Il motivo del perché usiamo le cache è perché, quando si esegue un programma, c'è la possibilità che il programma, in un qualsiasi momento, richieda alcune istruzioni immagazzinate vicino all'istruzione che sta venendo eseguita. Parliamo dunque di \textbf{località temporale} e \textbf{località spaziale}.
\nl
I dati sono spostati verso la cache in blocchi (o linee), proprio per il concetto di località. Inoltre, è molto più rapido per una cache fare un'unico spostamento di $n$ elementi piuttosto che fare $n$ spostamenti di un elemento ciascuno.
\nl
All'interno di una CPU ci sono 3 tipi di cache, cioè la cache \textbf{Level 1} (o \textbf{L1}), \textbf{Level 2} (o \textbf{L2}) e \textbf{Level 3} (o \textbf{L3}). La cache L1 è la più rapida, ma anche la più piccola, mentre invece la cache L3 è la più lenta, ma anche la più spaziosa. In base al tipo di cache, un dato potrebbe essere immagazzinato sia in una cache (come ad esempio la L1) che in quelle sottostanti, ma generalmente dipende da come viene implementata.
\nl
Quando la CPU richiede un dato (diciamo il valore di una variabile \verb|x|), allora quel dato verrà cercato nella cache L1; se non verrà trovato, la CPU cercherà poi il dato nella cache L2; se non venisse trovato cercherà poi nella cache L3 e infine, se non trovasse il dato in nessuna delle cache, lo cercherà nella memoria.
% inserisci esempio con moltiplicazione tra matrici dal GitHub del corso
\nl
Ma come possiamo ottimizzare l'uso delle cache? Di norma, un programmatore non ha accesso alle cache, e non può specificare quale dato va dove. Quindi l'unico modo per ottimizzare un programma è tramite l'uso di varie flags di ottimizzazione di \verb|gcc|.

\subsection{Coerenza delle cache}

Ma come funzionano le cache quando sono coinvolti più core in un programma? Supponiamo di avere un programma con due threads dove il thread 0 è assegnato al core 0, mentre il thread 1 è assegnato al core 1. Supponiamo inoltre che il core 0 abbia una variabile privata \verb|y_0| e che il core 1 abbia invece le variabili private \verb|y_1| e \verb|z_1|; consideriamo inoltre una variabile condivisa \verb|x = 2|. Consideriamo il seguente programma, che si diversifica in base al core:

\begin{center}
    \begin{tabular}{|c||c|c|}
        \hline
        Time & Core 0 & Core 1 \\
        \hline\hline
        0 & \verb|y_0 = x;| & \verb|y_1 = 3 * x;| \\
        \hline
        1 & \verb|x = 7;| & Istruzioni che non usano \verb|x| \\
        \hline
        2 & Istruzioni che non usano \verb|x| & \verb|z_1 = 4 * x;| \\
        \hline
    \end{tabular}
\end{center}

Ora, nella cache del core 0 abbiamo che \verb|y_0 = 2| e \verb|x = 7|, ma che cosa avremo nella cache del core 1? Siccome \verb|x| viene modificato solo nella cache del core 0, questa modifica non arriva alla cache del core 1. Questo vuol dire che nella cache del core 1 avremo \verb|y_1 = 6| e \verb|z_1 = 8|, e non \verb|y_1 = 6| e \verb|z_1 = 28|.
\nl
Ci sono vari modi per risolvere questo problema, e uno di questi è lo \textbf{snooping cache coherence}. L'idea di base è che i core sono connessi tutti tramite un bus comune, che permette a tutti i core di vedere quali sono i dati che passano tramite tale bus. Ciò viene sfruttato dalle cache come canale di broadcast, per cui ogni volta che una cache viene aggiornata da un core, allora anche gli altri cores lo sapranno. Tuttavia, questo metodo non viene ormai più usato, e questo perché ormai, con processori con 64/128 cores, effettuare un broadcast è costoso.
\nl
Un altro metodo, più efficace, è il \textbf{directory based cache coherence}, per cui c'è una una struttura di dati, chiamata \textbf{directory}, che segna lo stato di ogni linea della cache, simile a una bitmap. Quando una variabile viene modificata, allora viene anche modificata la directory, e questa modifica viene propagata anche ai controllori di ogni cache, che invalideranno il contenuto nelle cache di ogni core dove la variabile non è aggiornata.

\subsection{Organizzazione della memoria}

Generalmente la memoria di un sistema può essere organizzata in due modi: può essere o \textbf{Uniform Memory Access} (\textbf{UMA}) o \textbf{Non-Uniform Memory Access} (\textbf{NUMA}). Per l'organizzazione UMA, i core condividono la stessa memoria, mentre per i NUMA, ogni gruppo di core ha una sua sezione di memoria. Con un sistema NUMA, bisogna fare attenzione a dove vengono allocati i dati, perché se dei dati che servono a un core vengono allocati nella parte di memoria nella quale il core non può accedere, allora si creano problemi di latenza: accedere a una memoria "locale" è infatti più economico che accedere a una memoria "remota". Tramite l'uso della libreria \verb|numa.h| è possibile gestire la modalità di allocazione dei dati (in caso si può usare anche l'utilità da terminale \verb|numactl|).
\nl
\section{OpenMP}

Un altro framework/API per creare applicazioni a memoria condivisa è \textbf{OpenMP}, che garantisce, rispetto a PThreads, un'interfaccia più ad alto livello. Il sistema su cui viene eseguito un programma di OpenMP è visto come una collezione di cores, dove ogni core ha accesso alla memoria principale.
\nl
Rispetto a PThreads, OpenMP cerca di eseguire un codice in maniera parallela partendo da un programma sequenziale, ed è molto più semplice da applicare rispetto al trasformare un programma con PThreads. Per poter usare OpenMP, serve usare delle \textbf{pragma} e un compilatore che supporti la libreria: questo perché ci sono alcune direttive che devono essere eseguite al momento di compilazione. Segue la definizione di una pragma:

\begin{definition}{Pragma}
    Le \textbf{pragma} sono alcune \textbf{direttive} speciali che vengono \textbf{riconosciute} dal \textbf{preprocessore}. In \verb|C|, una pragma ha la forma \verb|#pragma|. Se un compilatore non supportasse le pragma di una libreria, allora le pragma verrebbero ignorate.
    \nl
    Le pragma possono usare delle \textbf{clausole}, ovverosia delle \textbf{direttive} che \textbf{modificano le pragma} e la loro esecuzione
\end{definition}

La pragma più utilizzata di OpenMP è \verb|#pragma omp parallel|, che dice al compilatore che il blocco di codice successivo va eseguito in parallelo. Per fare un esempio:

\begin{codeblock}{HelloWorld.c}
    \begin{lstlisting}[language = C]
#include <stdio.h>
#include <stdlib.h>
#include <omp.h>

void Hello(void) {
    int my_rank = omp_get_thread_num();
    int thread_count = omp_get_num_threads();

    printf("Hello dal thread %d su %d\n", my_rank, thread_count);
}

int main(int argc, char** argv[]) {
    int thread_count = strtol(argv[1], NULL, 10);

    #pragma omp parallel num_threads(thread_count)
    Hello();

    return 0;
}\end{lstlisting}
\end{codeblock}

Possiamo compilare il codice tramite il seguente comando:

\begin{terminal}
    \begin{lstlisting}[style = notexterm]
$ gcc -g -Wall -fopenmp -o hello_world HelloWorld.c
$ ./hello_world 4
    \end{lstlisting}
    \begin{tcolorbox}
        \begin{lstlisting}[basewidth=0.44em, numbers=none]
Hello dal thread 1 di 4
Hello dal thread 0 di 4
Hello dal thread 2 di 4
Hello dal thread 3 di 4\end{lstlisting}
    \end{tcolorbox}
\end{terminal}

Come possiamo impostare il numero di threads the eseguiranno il codice? Ci sono vari modi, che verranno ora elencati in ordine gerarchico (per primi quelli con gerarchia più bassa):
\begin{itemize}
    \item In modo \textbf{universale}: attraverso l'impostazione della variabile d'ambiente (enviromental variable di UNIX) \verb|OMP_NUM_THREADS|. Questa variabile può essere impostata attraverso il comando \verb|export| di UNIX:
    \begin{terminal}
        \begin{lstlisting}[style = notexterm]
$ echo $OMP_NUM_THREADS    # Per ottenere il valore
$ export OMP_NUM_THREADS=<num_threads>     # Per salvare il valore\end{lstlisting}
    \end{terminal}

    \item A livello di \textbf{programma}: attraverso l'uso della funzione \verb|omp_set_num_threads()| a inizio codice;
    \item A livello di \textbf{pragma}: usando la clausola \verb|num_threads()| nella pragma.
\end{itemize}

Per sapere quanti threads stanno eseguendo una certa parte di codice, possiamo usare la funzione \verb|omp_get_num_threads()| (se verrà eseguita in una porzione di codice sequenziale, ritornerà 1), mentre invece per avere il rank del thread bisognerà usare la funzione \verb|omp_get_thread_num()|.
\nl
Una differenza sostanziale con PThreads è che OpenMP potrebbe, se lo ritiene necessario, creare dei threads fin dall'inizio dell'esecuzione del codice e lasciarli in \verb|sleep| fino a che non servano, il che è sostanzialmente diverso da come funziona PThreads (può anche succedere, se una sezione di codice vada eseguita in sequenziale, che i threads vengano interrotti e messi in \verb|sleep| fino al momento in cui servano nuovamente).
\nl
Come abbiamo detto, non tutti i compilatori di C supportano OpenMP, e se un compilatore non supportasse la libreria potremmo avere problemi se provassimo a chiamare delle funzioni della libreria stessa. Come possiamo risolvere questo problema? Possiamo usare delle pragma specifiche per definire dei corsi d'azione diversi in base al fatto che il compilatore supporti OpenMP o meno:
\pagebreak
\begin{codeblock}{OpenMPCheck.c}
    \begin{lstlisting}[language = C]
// Inclusione delle librerie...
#ifdef _OPEN_MP
#include <omp.h>
#endif

void parallelFunc() {
    // Nella funzione che verrà eseguita in parallelo

    #ifdef _OPEN_MP
        int my_rank = omp_get_thread_num();
        int thread_count = omp_get_num_threads();
    #else
        int my_rank = 0;
        int thread_count = 1;
    #endif

    // Altro codice...
}

int main() {
    // Codice del main...
    
    return 0;
}\end{lstlisting}
\end{codeblock}

Come possiamo garantire invece ciò che su PThreads avevamo con le mutex? Si può usare la pragma \verb|#pragma omp critical|: tale pragma considererà il blocco sotto di essa come un blocco di codice che deve essere gestito da una mutex, e che quindi solo un thread alla volta potrà eseguire.
\nl
Possiamo anche usare un'altra pragma, che può essere usata solo in alcuni casi: la pragma \verb|#pragma omp atomic|. Ci sono alcune operazioni che le CPU supportano che sono definite come \textbf{atomiche}: queste istruzioni permettono di \textbf{eseguire molteplici istruzioni} di Assembly \textbf{come una singola istruzione}. Ad esempio, se avessimo un'istruzione di C tipo \verb|y += 1;|, allora le istruzioni che Assembly farebbe sarebbero generalmente \verb|LOAD|, \verb|ADD| e \verb|STORE|. Tuttavia, se l'addizione venisse supportata come operazione atomica dalla CPU, allora si potrebbe fare l'addizione, a livello Assembly, con una singola operazione. Un esempio segue:

\begin{codeblock}{Critical.c}
    \begin{lstlisting}[language = C]
int y = 0;      // Variabile pubblica per tutti i threads
#pragma omp critical
{   
    int my_rank = omp_get_thread_num();     // Variabile privata del thread
    y += my_rank;
}\end{lstlisting}
\end{codeblock}
\pagebreak
\begin{codeblock}{Atomic.c}
    \begin{lstlisting}[language = C]
// Tramite la pragma atomic sarebbe così:
int y = 0;

#pragma omp parallel
{
    int my_rank = omp_get_thread_num();

    #pragma omp atomic
    y += my_rank;
}\end{lstlisting}
\end{codeblock}

È importante far sì che l'istruzione che vogliamo eseguire atomicamente non richieda l'esecuzione di una funzione: se avessimo qualcosa tipo \verb|y += sum_trapezoid()|, e se \verb|sum_trapezoid()| richiedesse di fare vari calcoli che potrebbero comprendere all'interno sezioni critiche, allora la funzione verrebbe eseguita in parallelo da ogni thread senza controllo sulle sezioni critiche, e l'unica operazione atomica sarebbe la somma finale sulla variabile \verb|y|.

\subsection{Operazioni di riduzione}

Anche OpenMP permette di avere operazioni di riduzione, che vengono specificate nelle pragma che specificano quali blocchi di codice vadano eseguiti in parallelo. Ad esempio, supponiamo di avere una funzione \verb|local_sum(int a, int b, int n)|, e che dobbiamo far sommare in un'unica variabile \verb|result| tutti i risultati di \verb|local_sum()|. Potremmo fare qualcosa del genere:

\begin{codeblock}{Reductions.c}
    \begin{lstlisting}[langauge = c]
int result = 0;
#pragma omp parallel
{
    #pragma omp critical
    result += local_sum(a, b, n);
}\end{lstlisting}
\end{codeblock}

Tuttavia, in questo modo tutti i threads eseguirebbero la funzione in sequenziale, e questo non sarebbe efficiente. Se ci salvassimo il risultato di \verb|local_sum()| in una variabile privata, staremmo comunque facendo un'operazione di riduzione. Tuttavia, OpenMP ha un modo per definire delle operazioni di riduzione, tramite la clausola \verb|reduction(<operatore>: <variabile_critica>)|:

\begin{codeblock}{Reductions.c}
    \begin{lstlisting}[langauge = c]
int result = 0;
#pragma omp parallel reduction(+: result)
result += local_sum(a, b, n);\end{lstlisting}
\end{codeblock}

È importante specificare la clausola \verb|reduction|, altrimenti si avrebbero problemi di forza critica.