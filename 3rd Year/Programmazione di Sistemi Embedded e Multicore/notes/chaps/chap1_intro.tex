\chapter{Programmazione Parallela}

Ad oggi, molte tasks e molti programmi necessitano di grandi capacità di calcolo, soprattutto nel campo delle AI, e costruire centri di elaborazione sempre più grandi può non sempre costituire una soluzione. Certo, negli ultimi anni abbiamo assistito a una grande evoluzione dei microprocessori e delle loro potenze, ma non è abbastanza avere hardware sempre più potente, serve saperlo impiegare bene.
\\\\
La società Nvidia, produttrice di \textbf{GPU}s (Graphical Processing Units) è riuscita, in questi anni, a produrre schede grafiche contenenti sempre più milioni di transistors, rendendo una GPU un oggetto estremamente complicato.
\\\\
Come detto in precedenza, avere oggetti così potenti non è abbastanza, serve saper usare questi ultimi nel modo migliore possibile, ed è proprio questo l'obiettivo di questo corso.