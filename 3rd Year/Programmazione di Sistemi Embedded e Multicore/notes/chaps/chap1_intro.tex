\chapter{Programmazione Parallela \texorpdfstring{\&}{&} C}

Ad oggi, molte tasks e molti programmi necessitano di grandi capacità di calcolo, soprattutto nel campo delle AI, e costruire centri di elaborazione sempre più grandi può non sempre costituire una soluzione. Certo, negli ultimi anni abbiamo assistito a una grande evoluzione dei microprocessori e delle loro potenze, ma non è abbastanza avere hardware sempre più potente, serve saperlo impiegare bene.
\nl
La società Nvidia, produttrice di \textbf{GPU}s (Graphical Processing Units) è riuscita, in questi anni, a produrre schede grafiche contenenti sempre più milioni di transistors, rendendo una GPU un oggetto estremamente complicato ma anche estremamente potente e versatile. Per la legge di Moore, il \textbf{numero di transistors} che si trovano all'interno di un circuito \textbf{raddoppia} sempre ogni anno, tuttavia ci sono dei limiti fisici che non possono essere valicati. Inoltre, avere più transistors all'interno di un circuito non è sempre positivo: più transistors equivale sì a una prestazione più alta, ma equivale anche a quantità maggiori di calore generato, e grandi quantità di calore rischiano di creare incosistenza nel circuito.
\nl
Come detto in precedenza, avere oggetti così potenti non è abbastanza, serve saper usare questi ultimi nel modo migliore possibile, ed è proprio questo l'obiettivo di questo corso: cercare di \textbf{programmare} nel modo \textbf{quanto più efficiente possibile} dei circuiti che possano eseguire codice in parallelo.
\nl
Per farlo, non possiamo usare linguaggi troppo ad alto livello come Python o Java: benché siano molto famosi e semplici da scrivere, non garantiscono un buon controllo sull'hardware. Una buona scelta sono i linguaggi a basso livello come \textbf{C}, \textbf{C++} o \textbf{Rust}, che offrono un ottimo compromesso tra facilità di scrittura del codice ed efficienza (altrimenti dovremmo scrivere il tutto in Assembly, il che risulterebbe forse troppo tedioso).

\section{Da Sequenziale in Parallelo}

Immaginiamo di voler fare un programma seriale in C che conti tutti i valori in un intervallo specifico $\{ 1, \; 2, \; \dots, \; n\}$. Un modo semplice per farlo sarebbe sommare, attraverso un \verb|for| loop, tutti gli elementi da $1$ a $n$:

\begin{codeblock}{Sum.c}
    \begin{lstlisting}[language = c]
sum = 0;
for (i = 0; i < n; i++) {
	x = compute_next_value(i);
	sum += x;
}   \end{lstlisting}
\end{codeblock}

Tuttavia con valori di $n$ molto larghi possiamo ottenere tempi enormi (immaginiamo con $10^{16}$, o numeri di questa magnitudine). Per renderlo parallelo, immaginiamo di avere $p$ cores (dove $p \ll n$), dove ogni core esegue il seguente codice:

\begin{codeblock}{SumForEachCore.c}
    \begin{lstlisting}[language = c]
// Variabili private di ogni core
sum_core_i = 0;
first_i = ...;
last_i = ...;

for (value_i = first_i; value_i < last_i; value_i++) {
	x_i = compute_next_value(i);
	sum_core_i += x_i;
}   \end{lstlisting}
\end{codeblock}

Generalmente, si hanno 2 modi per scrivere del codice parallelo:

\begin{definition}{Task e Data Parallelism}
    Definiamo come \textbf{task parallelism} quando una task principale viene spezzata in $k$ sotto-tasks e distribuita ai vari $p$ cores, i quali portano avanti le loro tasks autonomamente.
    \nl
    Definiamo invece con \textbf{data parallelism} quando i dati vengono divisi in $k$ sottogruppi, che vengono affidati poi ai diversi $p$ cores, che eseguono tutti quanti la stessa operazione.
\end{definition}

Normalmente, se ogni core potesse eseguire codice indipendentemente, non sarebbe diverso scrivere codice seriale dallo scrivere codice parallelo. Tuttavia, come possiamo far sì che i core completino completare una task comune? Non hanno modo di passarsi dei dati in questo modo. Di base, ci deve essere una \textbf{comunicazione} tra i core, principalmente per i seguenti motivi:
\begin{itemize}
    \item \textbf{Comunicazione}: se ad esempio i core devono condividere dei dati;
    \item \textbf{Divisione del carico}: far sì che i cores possano lavorare assieme, così da bilanciare il carico di lavoro;
    \item \textbf{Sincronizzazione}: siccome ogni core può avere tempi di esecuzione diversi, è importante che si sincronizzino a tratti, così da non creare concorrenza.
\end{itemize}

Come si possono scrivere programmi paralleli? Ci sono 4 librerie che verranno viste:
\begin{itemize}
    \item \textbf{Message-Passing Interface} (MPI)$_{L}$;
    \item \textbf{Posix Threads} (PThreads)$_{L}$;
    \item \textbf{OpenMP}$_{L + C}$;
    \item \textbf{CUDA}$_{L + C}$
\end{itemize}

Dove $X_L$ indica che la voce è una libreria e $X_{L + C}$ indica che la voce è sia una libreria che un compiler.

\begin{definition}{Classificazione di sistemi paralleli}
    I sistemi paralleli sono classificabili in multiple categorie. Una prima categorizzazione è la seguente:
    \begin{itemize}
        \item a \textbf{memoria condivisa}: i cores hanno accesso alla stessa porzione di memoria, e si sincronizzano per quanto riguarda la posizione dei dati in memoria;
        \item a \textbf{memoria distribuita}: ogni core ha una sua memoria privata, e per scambiare dati con gli altri core c'è bisogno che un core mandi un messaggio agli altri cores.
    \end{itemize}

    Un'altra possibile modalità di categorizzazione è la seguente:
    \begin{itemize}
        \item \textbf{Multiple Instruction Multiple Data} (\textbf{MIMD}): ogni core ha una sua unità di controllo, può eseguire istruzioni diverse ed è indipendente dagli altri cores;
        \item \textbf{Single Instruction Multiple Data} (\textbf{SIMD}): i cores hanno una stessa unità di controllo (quindi o eseguono tutti la stessa istruzione o sono tutti in idle). Un esempio di sistema SIMD è la GPU.
    \end{itemize}
\end{definition}

\begin{center}
    \begin{tabular}{|c||c|c|}
        \hline
         & \makecell{\textbf{Shared Memory}} & \textbf{Distributed Memory} \\
        \hline \hline
        \textbf{SIMD} & \makecell{CUDA$_{L + C}$} & \\
        \hline
        \makecell{\textbf{MIMD}} & \makecell{Posix Threads (PThreads)$_L$ \\ OpenMP$_{L + C}$ \\ CUDA$_{L + C}$} & Message-Passing Interface (MPI)$_{L}$ \\
        \hline
    \end{tabular}
\end{center}

\subsection{Concorrenza, parallelismo e distribuzione}

Non c'è un'unica definizione, ma generalmente noi possiamo dire che:
\begin{itemize}
    \item un sistema in concorrenza permette di eseguire tasks diverse ad un dato momento $t$ di esecuzione del sistema (ad esempio, il sistema operativo);
    \item un sistema in parallelo permette di eseguire multiple tasks (anche diverse) per risolvere un problema comune;
    \item un sistema distribuito permette di distribuire le tasks su diverse entità, che però potrebbero (come non potrebbero) collaborare.
\end{itemize}

 Generalmente, la differenza tra un sistema distribuito e un sistema parallelo si vede nella collaborazione del sistema, tuttavia il confine è molto labile.


\subsection{Review dell'architettura di Von Neumann}

L'architettura di Von Neumann è caratterizzata dalla presenza dei seguenti elementi:
\begin{itemize}
    \item \textbf{Memoria}: insieme di posizioni (locations), ognuna con un indirizzo (address). In una posizione ci possono essere o dati o istruzioni;
    \item \textbf{CPU} o \textbf{processore}: un'unità di controllo che decide quali istruzioni eseguire e una sezione di \textbf{datapath} che esegue le istruzioni. In una CPU, lo stato delle variabili è salvato in dei registri, ovverosia un tipo di memoria estremamente veloce (utilizza flip-flops); un registro importante è il \textbf{program counter} (\textbf{PC}), che salva l'indirizzo della prossima istruzione da eseguire;
    \item \textbf{Interconnect}: è il mezzo attraverso il quale la CPU scambia dati con la memoria. Generalmente è un bus, ma può essere anche di un altro tipo (a volte più complesso del semplice bus).
\end{itemize}

Le architetture di Von Neumann hanno un \textbf{bottleneck}, dato dal necessario spostamento dei dati: accedere a un dato in una cache di tipo $L_1$ (una cache molto veloce situata vicino ai core della CPU) potrà forse costare $0.5ns$, tuttavia portare un dato dalla memoria ai registri costa $100ns$. La maggior parte del tempo speso in questo tipo di architetture, viene speso nel suddetto spostamento di dati.
\nl
Un modo per mitigare questi problemi è usare le \textbf{cache}, che permettono di salvare temporaneamente dei dati. Più è recente un dato, più è probabile trovarlo nella cache. Programmare una cache è molto complicato, ma è possibile e se ben fatto permette di ridurre drasticamente i tempi di accesso in memoria.

\section{Il linguaggio C}

Scritto nel 1972 come linguaggio di programmazione che sostituisse Assembly, C è un linguaggio a basso livello che permette di avere un controllo granulare delle risorse usate, e che viene usato spesso per disegnare e creare software che deve essere veloce e performante. In C, ogni istruzione è scritta nel formato \verb|istruzione;|, dove il \texttt{;} separa un'istruzione da quella successiva. Come molti linguaggi, anche in C è possibile usare commenti (attraverso i simboli \verb|//| per il commento a linea singola e \verb|/* ... */| per il commento a multiple linee).
\nl
Per poter includere alcune funzioni di input e output, bisogna includere un file header. Il file responsabile per l'I/O è \verb|stdio.h|, e va incluso con il comando \verb|#include <stdio.h>|. L'inclusione di librerie esterne avviene in un momento di pre-processing, dove anche avviene ad esempio la rimozione dei commenti dai file di codice. Tale sessione di pre-processing viene fatta da un'entità chiamata \textbf{pre-processore}.
\nl
Ogni programma di C \textbf{deve} avere una funzione \verb|main|, ma può anche avere più funzioni. L'importante è che ci sia una funzione \verb|main|, poiché è da essa che parte l'esecuzione del codice. La funzione \verb|main| peraltro \textbf{deve} finire con l'istruzione \verb|return 0;|, poiché essa determina che il programma è stato eseguito con successo. Segue un esempio di programma in C:

\begin{codeblock}{HelloWorld.c}
    \begin{lstlisting}[language = c]
#include <stdio.h>

int main () {
    printf("First C program!");
    return 0;
}\end{lstlisting}
\end{codeblock}

La funzione \verb|printf()| fa parte della libreria \verb|stdio|, che è stata inclusa in testa al file.

\subsection{Variabili, \texttt{printf} e \texttt{scanf}}

In C si possono definire delle variabili, e assegnargli un valore, grazie alla seguente sintassi: \verb|<tipo variabile> <nome variabile> = <valore variabile>;|. Seguono degli esempi:

\begin{codeblock}{Variables.c}
    \begin{lstlisting}[language = c]
int a = 0;        // Dichiarazione di un intero (int)
float b = 4.65;   // Dichiarazione di un decimale a virgola mobile (float)
char c = 'c';     // Dichiarazione di un carattere (char)

int x, y;         // Dichiarazione di due interi senza aver 
                  // assegnato un valore
x = 4;            // Assegnamento di un valore alla variabile x
y = 5;            // Ugualmente alla riga precedente, y riceve un valore\end{lstlisting}
\end{codeblock}

Nell'ultima riga, abbiamo solo definito due variabili, ma non gli abbiamo assegnato alcun valore. Questa cosa è possibile in C, e torna utile quando dobbiamo definire una variabile ma non sappiamo ancora quale valore preciso dovrà essere inserito.
\nl
C permette anche di eseguire operazioni attraverso certi operatori. Segue una lista di tutte le operazioni disponibili:
\begin{center}
    \begin{tabular}{|c|c|c|}
        \hline
        \textbf{Operatore} & \textbf{Operazione} & \textbf{Sintassi} \\
        \hline
        \verb|+| & Somma & \verb|a + b| \\
        \verb|-| & Differenza & \verb|a - b| \\
        \verb|*| & Moltiplicazione & \verb|a * b| \\
        \verb|/| & Divisione (intera) & \verb|a / b| \\
        \verb|%| & Modulo (resto della divisione) & \verb|a % b| \\
        \verb|(| \verb|)| & Parentesi & \verb|(a + b) * c| \\
        \verb|++| & Incremento (di 1) & \verb|a++| \\
        \verb|--| & Decremento (di 1) & \verb|a--| \\
        \hline
    \end{tabular}
\end{center}

La funzione \verb|printf| accetta molteplici argomenti, e permette anche di stampare il valore delle variabili a schermo:

\begin{codeblock}{Printf.c}
    \begin{lstlisting}[language = c]
// Uso di printf con molteplici argomenti
printf("String 1");
printf("String 1", "String 2");
printf("String 1", ..., "String n");

// Stampa di variabili
int x = 4;
printf("Value of x: %d", x);\end{lstlisting}
\end{codeblock}

Si può usare il carattere \verb|%| seguito da una lettera per determinare il placeholder per la variabile che vuole essere stampata. Segue una lista dei possibili placeholders:

\begin{center}
    \begin{tabular}{|c|c|}
        \hline
        \textbf{Placeholder} & \textbf{Tipo variabile} \\
        \hline
        \verb|%d| & Interi (\verb|int|) \\
        \verb|%f| & Decimali a virgola mobile (\verb|float|) \\
        \verb|%c| & Caratteri (\verb|char|) \\
        \verb|%s| & Stringhe \\
        \verb|%u| & Interi senza segno \\
        \verb|%ld| & Interi lunghi (32 bits) \\
        \verb|%lu| & Interi lunghi senza segno (32 bits) \\
        \verb|%lld| & Interi grandi (64 bits) \\
        \verb|%llu| & Interi grandi senza segno (64 bits) \\
        \verb|%x| & Interi esadecimali con lettere minuscole \\
        \verb|%X| & Interi esadecimali con lettere maiuscole \\
        \verb|%o| & Interi ottali \\
        \verb|%e| & Notazione scientifica con lettere minuscole \\
        \verb|%E| & Notazione scientifica con lettere maiuscole \\
        \verb|%g| & Rappresentazione ristretta di \verb|%f| e \verb|%e| \\
        \verb|%G| & Rappresentazione ristretta di \verb|%f| e \verb|%E| \\
        \verb|%%| & Stampa di un simbolo \verb|%| \\
        \hline
    \end{tabular}
\end{center}

La funzione \verb|scanf| permette di ricevere dall'utente in input dei valori. La sintassi della funzione è \verb|scanf("<descrizione>", <lista indirizzi in memoria>)|. Nella descrizione devono figurare i placeholder con i corretti tipi di dati che vogliamo richiedere. Un esempio è:

\begin{codeblock}{PrintfWithPointers.c}
    \begin{lstlisting}[language = c]
int x;

// Il simbolo & ritorna l'indirizzo in memoria della variabile x,
// che specifica dove vogliamo salvare i dati raccolti con scanf
int* address = &x;

// Chiamando scanf, chiediamo all'utente di inserire un valore intero
scanf("Inserisci un intero: %d", address)
    \end{lstlisting}
\end{codeblock}

Nella sezione sui puntatori, verrà spiegato meglio il concetto di indirizzo di memoria.
\nl
Come in tutti gli altri linguaggi di programmazione, C segue uno specifico ordine di esecuzione: tutte le istruzioni vengono eseguite in ordine sequenziale, una dopo l'altra. Supponiamo di avere queste righe di codice:

\begin{codeblock}{ExecutionOrder.c}
    \begin{lstlisting}[language = c]
int x = 40;
int y = 60;
y = y + x;
x = x - (y / 2);
    \end{lstlisting}
\end{codeblock}

Il risultato finale di \verb|x| sarà \verb|-10|, poiché prima gli viene assegnato \verb|40|, e poi gli viene sottratto il valore di \verb|(y + x) / 2|.

\subsection{Tipi primitivi, promozione e casting}

Un computer non può memorizzare numeri infinitamente grandi o piccoli, in quanto si ha una precisione e una memoria limitata. Questo comporta che ci sono vari tipi di numeri, che possono occupare più o meno spazio in memoria rispetto ad altri. Questi tipi di variabili sono detti \textbf{tipi primitivi}, poiché sono definiti da C stesso. Una lista di tipi primitivi è la seguente:

\begin{center}
    \begin{tabular}{|c|c|}
        \hline
        \textbf{Tipo primitivo} & \textbf{Spazio in memoria occupato} \\
        \hline
        \verb|char| & 8 bits \\
        \verb|short| & 16 bits \\
        \verb|int| & 32 bits \\
        \verb|long| & 32 bits \\
        \verb|long long| & 64 bits \\
        \verb|float| & 32 bits \\
        \verb|double| & 64 bits \\
        \verb|long double| & 128 bits \\
        \hline
    \end{tabular}
\end{center}

Esistono anche delle versioni \verb|unsigned| dei primi 5 tipi, che, come suggerisce il nome, sono privi di segno, e possono essere dunque solo positivi.
\nl
Quando si esegue un'operazione tra due variabili di tipo diverso, C effettua una conversione della variabile con tipo più piccolo (a livello di bits) a una variabile di tipo più grande. Questa conversione è chiamata \textbf{promozione}.
\nl
Nel caso si volesse cambiare in corso d'opera il tipo di una variabile, si può effettuare il \textbf{casting}, ovverosia il cambio di tipo. Per esempio, supponiamo di avere due interi e volerne fare la divisione. Un modo per sfruttare il casting sarebbe il seguente:

\begin{codeblock}{CastingAndPromotion.c}
    \begin{lstlisting}[language = c]
int x, y;
x = 10;
y = 3;
z = (float) x / y;\end{lstlisting}
\end{codeblock}

\subsection{Logica e cicli}

Quando si vuole eseguire un'azione piuttosto che un altra, in base ad una certa condizione, si possono usare vari controlli condizionali. Ci sono due tipi di controlli condizionali: l'\verb|if-else if-else| e lo \verb|switch|. L'\verb|if-else if-else| è strutturato in questo modo:

\begin{codeblock}{IfElse.c}
    \begin{lstlisting}[language = c]
if (<condizione>) {
    // Codice...
} else if (<condizione>) {
    // Codice...
} else {
    // Codice...
}\end{lstlisting}
\end{codeblock}

Lo \verb|switch| invece è strutturato nel seguente modo:

\begin{codeblock}{Switch.c}
    \begin{lstlisting}[language = c]
switch (<variabile>) {
    case <valore>:
        // Codice...
        break;
    case <valore>:
        // Codice...
        break;
    // Tanti case quanti sono i casi che vogliamo gestire...

    default:
        // Codice...
        break;
}\end{lstlisting}
\end{codeblock}

Sia l'\verb|if| che lo \verb|switch| si avvalgono delle operazioni logiche per funzionare. Tali operazioni sono basate sul fatto che una condizione a un certo punto sia o vera o falsa. Infatti, esistono i valori booleani: \verb|true| e \verb|false|.
\nl
Per poter creare espressioni logiche, possiamo usare i seguenti operatori:

\begin{center}
    \begin{tabular}{|c|c|c|}
        \hline
        \textbf{Operatore} & \textbf{Nome} & \textbf{Sintassi} \\
        \hline
        \verb|<| & Minore & \verb|a < b| \\
        \verb|>| & Maggiore & \verb|a > b| \\
        \verb|<=| & Minore o uguale & \verb|a <= b| \\
        \verb|>=| & Maggiore o uguale & \verb|a >= b| \\
        \verb|==| & Uguale & \verb|a == b| \\
        \verb|!=| & Diverso & \verb|a != b| \\
        \verb|&&| & E (AND) & \verb|a && b| \\
        \verb+||+ & O (OR) & \verb|a || b| \\
        \verb|!| & Non (NOT) & \verb|!a| \\
        \hline
    \end{tabular}
\end{center}

Altri strumenti che usano la logica sono i cicli, che permettono di eseguire codice fino a che una certa condizione non si avvera / finché una certa condizione non diventa falsa. I tre cicli che ci sono in C sono il \verb|for|, il \verb|while| e il \verb|do-while|.
\nl
Il \verb|for| ripete un'operazione per un numero finito di volte, fino a che la condizione non diventa falsa. La cosa comoda del ciclo \verb|for| è che permette di tener conto dell'iterazione in cui ci si trova.

\begin{codeblock}{ForLoop.c}
    \begin{lstlisting}[language = c]
for (inizializzazione ciclo; condizione ciclo; incremento ciclo) {
    // Codice...
}

// Ad esempio...
for (int i = 0; i < 10; i++) {
    // Codice...
}\end{lstlisting}
\end{codeblock}

Il ciclo \verb|while| e il \verb|do-while| sono simili concettualmente: entrambi eseguono il codice al loro interno fino a che la condizione iniziale non diventa falsa. La differenza fra i due è che il \verb|do-while| esegue almeno una volta il codice all'interno del blocco \verb|do|, mentre il \verb|while| non assicura che il codice al suo interno venga eseguito.

\begin{codeblock}{DoAndDoWhile.c}
    \begin{lstlisting}[language = c]
// Ciclo while
while (<condizione>) {
    // Codice...
}

// Ciclo do-while
do {
    // Codice...
} while (<condizione>)\end{lstlisting}
\end{codeblock}

Si possono usare anche due keywords importanti: \verb|break| farà uscire il codice dal ciclo in cui si trova, in modo anticipato (ad esempio se una sotto-condizione viene raggiunta); \verb|continue| farà saltare un'iterazione del ciclo.

\begin{exercise}
    Si scriva un programma che legge in input due numeri interi.
    \begin{itemize}
        \item [1)] Il programma deve stampare se il primo numero è maggiore, minore o uguale al secondo.
        \item [2)] Il programma deve stampare la somma e il prodotto dei due numeri.
    \end{itemize}
\end{exercise}

La soluzione sarebbe la seguente:

\begin{codeblock}{Exercise1\_2\_0.c}
    \begin{lstlisting}[language = c]
#include <stdio.h>

int main () {
    int a, b;
    printf("Inserisci il tuo primo numero a: ");
    scanf("%d", &a);
    printf("Inserisci in tuo secondo numero b: ");
    scanf("%d", &b);

    if (a < b) {
        printf("%d è maggiore di %d\n", a, b);
    } else if (a == b) {
        printf("%d è uguale a %d\n", a, b);
    } else {
        printf("%d è minore di %d\n", a, b);
    }

    int sum = a + b;
    int product = a * b;
    printf("La somma è uguale a %d\nIl prodotto è uguale a %d", sum, product);
    return 0;
}\end{lstlisting}
\end{codeblock}

\begin{exercise}
    Si scriva un programma che legge in input un numero intero \verb|a|.
    \begin{itemize}
        \item [1)] Il programma stampa tutti i numeri da 0 ad \verb|a| in ordine crescente.
        \item [2)] Il programma stampa tutti i numeri da 0 ad \verb|a| in ordine decrescente.
    \end{itemize}
    Per svolgere l'esercizio si utilizzino cicli \verb|for|
\end{exercise}

La soluzione sarebbe la seguente:

\begin{codeblock}{Exercise1\_2\_1.c}
    \begin{lstlisting}[language = c]
#include <stdio.h>

int main () {
    int a;
    printf("Inserisci il tuo numero a: ");
    scanf("%d", &a);

    for (int i = 0; i <= a; i++) {
        printf("%d\n", i);
    }

    for (int i = a; i <= 0; a--) {
        printf("%d\n", a);
    }

    return 0;
}\end{lstlisting}
\end{codeblock}

\begin{exercise}
    Si scriva un programma che legge in input un numero intero \verb|a|.
    \begin{itemize}
        \item [1)] Il programma stampa tutti i numeri da 0 ad \verb|a| in ordine crescente.
        \item [2)] Il programma stampa tutti i numeri da 0 ad \verb|a| in ordine decrescente.
    \end{itemize}
    Per svolgere l'esercizio si utilizzino cicli \verb|while|
\end{exercise}

La soluzione sarebbe la seguente:
\pagebreak
\begin{codeblock}{Exercise1\_2\_2.c}
    \begin{lstlisting}[language = c]
#include <stdio.h>

int main () {
    int a, i;
    printf("Inserisci il tuo numero a: ");
    scanf("%d", &a);
    i = 0;

    while (i <= a) {
        printf("%d\n", i);
        i++;
    }

    i = a;
    while (i >= 0) {
        printf("%d\n", i);
        i--;
    }

    return 0;
}\end{lstlisting}
\end{codeblock}

\subsection{Array, stringhe e matrici}

Un array è una \textbf{collezione ordinata} di elementi dello \textbf{stesso tipo}, e ogni array richiede che venga specificata la grandezza quando l'array viene dichiarato. La sintassi per creare un array è 
\begin{center}
    \verb|<tipo> <nome_array> [<grandezza>]|
\end{center}

Ad esempio, un array di 3 interi viene specificato con \verb|int my_array [3]|. Per stampare il contenuto di un array non possiamo passare direttamente la variabile dell'array a \verb|printf()|, ma bisogna usare un \verb|for| loop per iterare attraverso l'intero array. Se passassimo un array alla funzione \verb|printf()| non potremmo predirre direttamente il risultato di tale funzione.
\nl
Si possono anche creare \textbf{array multidimensionali}, per poter rappresentare tabelle o matrici. La sintassi per creare un array multidimensionale è la stessa di Python, dove la sintassi è
\begin{center}
    \verb|<tipo> <variabile> [<grandezza righe>][<grandezza colonne>]|
\end{center}

In linguaggi più ad alti livelli, le stringhe sono considerate come un tipo a sé stante, ma in C questo non è il caso. Infatti, queste sono viste come \textbf{array di caratteri}. Ipotizziamo quindi di voler immagazzinare in una stringa di C la frase "\textit{Hello world}", come lo faremmo in C? Semplicemente scrivendo
\begin{center}
    \verb|char string[12] = "Hello world";|
\end{center}

Ma perché abbiamo dato una lunghezza di 12 caratteri e non 11? Perché al termine di ogni stringa c'è un carattere chiamato \textbf{terminatore}, identificato con \verb|\0|. Dunque nella memoria avremmo i seguenti caratteri:

\begin{center}
    \begin{tabular}{|c|c|c|c|c|c|c|c|c|c|c|c|}
        \hline
        \verb|H| & \verb|e| & \verb|l| & \verb|l| & \verb|o| & & \verb|w| & \verb|o| & \verb|r| & \verb|l| & \verb|d| & \verb|\0| \\
        \hline
    \end{tabular}
\end{center}

Per eseguire operazioni particolari sulle stringhe, possiamo anche usare la libreria \verb|string.h|. Alcune delle funzioni più utilizzate sono \begin{itemize}
    \item \verb|strlen(<stringa>)|: per ottenere la lunghezza di una stringa
    \item \verb|strcpy(<destinazione>, <source>)|: per copiare una stringa;
    \item \verb|strcat(<destinazione>, <source>)|: per concatenare la stringa definita da \verb|<source>| nella stringa definita da \verb|<destinazione>|;
    \item \verb|strcmp(<stringa_1>, <stringa_2>)|: per comparare due stringhe.
\end{itemize}

\subsection{Puntatori \texorpdfstring{\&}{&} gestione della memoria}

In C, è possibile accedere direttamente all'indirizzo di memoria dove è salvata una variabile. Per farlo, bisogna usare la sintassi 

\begin{center}
    \verb|<tipo variabile>* <nome variabile> = &<nome variabile>|
\end{center}

Ci sono due operatori che possiamo usare assieme ai puntatori, ovverosia \verb|*| e \verb|&|:
\begin{itemize}
    \item l'operatore \verb|*| restituisce il contenuto della cella di memoria indicata dal puntatore;
    \item l'operatore \verb|&| restituisce il puntatore di una variabile.
\end{itemize}

La memoria può anche essere gestita liberamente dal programmatore in C. Tuttavia va fatta una distinzione tra due parti della memoria diverse, ovverosia lo stack e l'heap.

\begin{definition}{Stack \& Heap}
    La memoria di un computer è divisa in 2 tipi:
    \begin{itemize}
        \item \verb|STACK|: è una parte \textbf{statica} della memoria, che viene usata per l'esecuzione delle funzioni;
        \item \verb|HEAP|: è una parte \textbf{dinamica} della memoria, che permette di allocare oggetti ed elementi di dimensione variabile. In C, va allocata manualmente;
    \end{itemize}
\end{definition}

L'allocazione di memoria viene fatta tramite l'uso della libreria \verb|stdlib.h|, che comprende principalmente 4 funzioni:
\begin{itemize}
    \item \verb|void* malloc(int size)|: permette di allocare un blocco di memoria contiguo, il cui spazio totale è determinato da \verb|size|;
    \item \verb|void* realloc(void* ptr, int size)|: realloca a memoria localizzata dal puntatore \verb|ptr| a una dimensione determinata da \verb|size|. Questa funzione può essere usata quando si necessita di riallocare in memoria una sezione di spazio contiguo più grande dello spazio originariamente richiesto. Non sempre è possibile estendere un blocco di memoria: potrebbe essere necessario che il sistema operativo riallochi tutto il blocco di memoria in un secondo spazio. Infatti, se questo dovesse essere il caso, verrà ritornato un puntatore nuovo;
    \item \verb|void* calloc(int numElements, int size)|: permette di riservare una porzione di memoria di dimensione uguale a \verb|numElements * size|, dove \verb|size| è la dimensione dell'elemento singolo che vuole essere allocato;
    \item \verb|void* free(void* ptr)|: libera la memoria, indicata dal puntatore \verb|ptr|, riservata precedentemente con una delle precedenti funzioni.
\end{itemize}

È molto importante ricordarsi di liberare memoria quanto più possibile, proprio perché C non è dotato di un garbage collector, ed è compito del programmatore ricordarsi di liberare la memoria quando necessario.
\nl
Se serve sapere quanti bytes serve allocare per certi oggetti o elementi, potremmo voler usare la funzione \verb|sizeof()|. Tale funzione restituisce, in bytes, la dimensione dell'elemento. Questa funzione è comoda perché non sempre possiamo sapere con precisione il numero di bytes necessari per allocare qualcosa.