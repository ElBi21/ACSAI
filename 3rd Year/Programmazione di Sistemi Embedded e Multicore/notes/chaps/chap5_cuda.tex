\chapter{Programmazione di GPU e CUDA}

Una componente essenziale in tutti i computer è la CPU, che è nata essenzialmente per eseguire calcolo sequenziale nel più breve tempo possibile. Fin dalla loro creazione, l'idea era di ottimizzare quanto più possibile l'efficienza e la riduzione della latenza. Un computer necessita di una CPU per poter funzionare, ma non necessita strettamente invece di una GPU.

\begin{definition}{Graphics Processing Unit}
    Una \textbf{Graphics Processing Unit} (\textbf{GPU}) è una componente ideata per condurre tasks di grafica, ideata per eseguire operazioni in parallelo
\end{definition}

La struttura di una GPU è molto diversa da quella di una CPU: quest'ultima infatti è pensata per eseguire codice sequenziale, non parallelo. Una GPU è composta dalle seguenti componenti:
\begin{itemize}
    \item una sezione ridotta di cache, che puntano a massimizzare il throughput della memoria;
    \item una sezione massiva di ALUs (più lenti di quelli nelle CPU), che però sono collegati da una pipeline pemsata apposta per svolgere compiti in parallelo;
    \item una memoria di tipo HBM (High Bandwidth Memory) chiamata VRAM (\textbf{Video RAM}).
\end{itemize}

Su questo corso ci concentreremo sulle architetture delle GPU CUDA, create da NVIDIA. In questo tipo di architettura ci sono vari cores, chiamati \textbf{streaming processors} (\textbf{SP}), che sono uniti in più blocchi, chiamati \textbf{streaming multiprocessors} (\textbf{SM}). Ogni streaming multiprocessor è controllato da una stessa unità di controllo: questo vuol dire che ogni streaming processor in uno stesso multiprocessor eseguirà la stessa istruzione di Assembly in un dato momento.
\nl
Ci sono alcuni caveat quando si programmano le GPU: la memoria della GPU e quella della CPU sono diverse, e quindi per poter usare un dato sulla GPU che risiede nella memoria della CPU, questo dato andrà prima caricato sulla memoria della GPU.

\section{CUDA}

Abbiamo spesso parlato di CUDA, ma di cosa parliamo quando usiamo questo termine? CUDA è un acronimo che sta per \textbf{Compute Unified Device Architecture}, che intende generalmente l'architettura delle schede GPU di NVIDIA. CUDA offre due API diverse, una più low level e una più high level per i 3 grandi sistemi operativi.
\nl
CUDA va oltre il semplice linguaggio di programmazione: è una \textbf{piattaforma di calcolo} a tutti gli effetti. Le piattaforme CUDA vengono considerate come \textbf{co-processori} della CPU, non come sostituzioni. Infatti, per funzionare, la GPU necessita della CPU.
\nl
Al giorno d'oggi ci sono due grandi competitors nel mercato delle GPU: NVIDIA e AMD. Al momento, NVIDIA è la società più quotata, ma anche AMD offre hardware performante. ADM offre un'API proprietara chiamata HIP, che nella maggior parte delle istruzioni è simile a CUDA. Ci sono anche altre due API open source, chiamate OpenCL e OpenACC.
\nl
CUDA organizza i suoi thread in una struttura a sei dimensioni: ogni thread è posizionato in un blocco che va da una a tre dimensioni, e ogni blocco è posizionato in una griglia che va anch'essa da una a tre dimensioni. Dunque ogni thread può sapere la sua posizione in questo spazio a sei dimensioni, chiamando opportune funzioni offerte dall'API di CUDA.
\nl
Ogni piattaforma CUDA ha una sua capacità di calcolo, che viene chiamata \textbf{CUDA compute capability}. La capacità viene rappresentata da un numero di versione chiamato \textbf{SM version}.

\subsection{Scrivere un programma di CUDA}

La struttura di un programma CUDA è la seguente:
\begin{itemize}
    \item [1)] \textbf{allocare la memoria} sulla piattaforma CUDA;
    \item [2)] \textbf{trasferire i dati} dalla DRAM alla VRAM della piattaforma;
    \item [3)] \textbf{eseguire un kernel CUDA}, ovverosia una funzione;
    \item [4)] una volta terminato il kernel, serivrà \textbf{copiare i dati} processati dalla VRAM della piattaforma nella DRAM.
\end{itemize}

Per scrivere un programma di CUDA bisogna specificare una funzione, chiamata \textbf{kernel}, che andrà eseguita su tutti i threads. Chiaramente non è pensabile specificare, thread per thread, cosa ogni thread dovrà eseguire, qundi l'idea è quella di specificare come arrangiare i threads in blocchi e griglie. Di norma, la GPU ha già una sua struttura di blocchi e griglie, tuttavia noi abbiamo la possibilità di scegliere una nostra struttura che sia compatibile con l'architettura della nostra GPU.
\nl
Per eseguire una funzione su una GPU NVIDIA, servirà specificare appunto la dimensione dei blocchi e della griglia, e ciò si fa attraverso le seguenti linee di codice:

\begin{codeblock}{CUDAStartKernel.cu}
    \begin{lstlisting}[language = C]
#include <stdio.h>
#include <cuda.h>

__global__ void hello() {
    printf("Hello world");
}

int main() {
    dim3 block(3, 2);
    dim3 grid(4, 3, 2);
    hello <<<grid, block>>>();
    cudaDeviceSynchronize();
    return 0;
}

my_function<<<grid, block>>>();\end{lstlisting}
\end{codeblock}