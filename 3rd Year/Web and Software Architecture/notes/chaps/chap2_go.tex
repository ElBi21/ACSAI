\chapter{Backend and Go}

Go is a statically typed programming language created by Google, and it's made for building high-performance and scalable applications. Widely used in the Web, it allows to build powerful backends with high-level features such as \textbf{garbage collection} and \textbf{memory safety}. Its syntax is similar to the one of C, but it's way simpler. It's also more verbose though, so it's very similar to the choice that also Rust's creators took.
\nl
Go is structured in packages, where each package represents a collection of files and folders. A package is declared with the syntax \verb|package <name>|. In a folder, only one package may be declared. Go requires a \verb|main| package, which will contain the first functions to be executed.
\nl
Within a package, all the declarations are \textbf{global}: this means that a function declared in a file can be used in another file within the same package. In order to use other packages, we use the \verb|import <package_name>| syntax. Let's see an example of Go code for a simple Hello World program:

\begin{codeblock}{HelloWorld.go}
    \begin{lstlisting}[language = go]
package main

// Import the fmt (format) package for including functions like "Println()"
import "fmt"

func main() {
    fmt.Println("Hello World")
}\end{lstlisting}
\end{codeblock}

We can now build our code (because Go produces a binary) with the following command:

\begin{terminal}
    \begin{lstlisting}[style = notexterm]
go build HelloWorld.go -o HelloWorld
    \end{lstlisting}
\end{terminal}

and then run it as a standard executable. By executing the binary, we obtain the following:

\begin{terminal}
    \begin{lstlisting}[style = notexterm]
 $ ./HelloWorld
    \end{lstlisting}
    \begin{tcolorbox}
        \begin{lstlisting}[basewidth=0.44em, numbers=none]
Hello World\end{lstlisting}
    \end{tcolorbox}
\end{terminal}

Go is a statically typed language, so we are encouraged to write down the types of the variables that we create for code security. Variables are defined either through the \verb|var| keyword or through the \verb|:=| operator.
\nl
Some types that are allowed in Go are \verb|bool| for booleans, \verb|intXX| and \verb|uintXX| for respectively signed and unsigned integers (the \verb|XX| stands for the number of bits that the number should take, so a value between \verb|8|, \verb|16|, \verb|32| and \verb|64|), \verb|string| for strings, \verb|byte| for the bytes (it's an alias for \verb|uint8|), \verb|rune| for Unicode characters (has a size of \verb|32| bits), \verb|uintptr| for memory pointers, and so on and so forth...
\nl
Each variable is initialized to a specific value if no value is assigned. For most of the types, the standard value is 0, for strings, it's an empty string and for booleans it's \verb|false|. Inference is also done if a type is missing. We can also declare constants if we want to. Here is an example of some variables:

\begin{codeblock}{Variables.go}
    \begin{lstlisting}[language = go]
// Create the variable without assigning a value
var anInteger int

// Assignment of the value
anInteger = 6

// Inference: the type will be "float"
var pi = 3.14

// If we are not sure of how many bits we need, we can just use int or uint
var anotherInteger uint = 14

// Cast the second integer to an integer of type int64
var mySum int64 = 5 + int64(anotherInteger)

// Use the := notation instead of "var"
eulernumber := 2.7183

// We can also declare constants with the "const" keyword
const g float32 = 9.81\end{lstlisting}
\end{codeblock}

We can also declare arrays which, similarly to C's arrays, all have a fixed length. Arrays are declared with the \verb|var <name> [<size>]<type>| syntax. This syntax underlines how Go's arrays can store only one type of elements. So, for instance:
\pagebreak
\begin{codeblock}{Arrays.go}
    \begin{lstlisting}[language = go]
var myArray [10]int

// If we want to also assign some values, we have to do the following:
assignedArray := [10]int{1, 2, 3, 4, 5, 6, 7, 8, 9, 10}

// We can access to an array's value with the "[]" notation
var mySum = assignedArray[5] + assignedArray[3]
    \end{lstlisting}
\end{codeblock}

Go also uses the concept of slices, which allows to access to portions of the array just like in Python. The syntax is \verb|[initial index (included):final index(excluded)]|, wehre the lowest initial index is \verb|0| and the highest final index is the length of the array. Slices do not contain data, but just point to a section of the array.
\nl
There is a way to have dynamic arrays, and it's to use the slices mechanism. For instance, with \verb|bools := []bool{true, false, false}| creates a dynamic array. But why does it use the slices? Because with the previous notation we are just creating an array of not known size and then we created a slice to that array.
\nl
Slices have also two important properties: \textbf{length} and \textbf{capacity}.

\begin{definition}{Length and Capacity of a slice}
    The \textbf{length} of a slice denotes the number of elements in the slice.
    \nl\nl
    The \textbf{capacity} of a slice denotes the number of elements that the sliced array contains.
\end{definition}

Let's make a couple examples, to better grasp the concept: if we define an empty slice with \verb|var a_slice []int|, then we have a slice with both length and capacity equal to \verb|0|. Now though, if we append one element with \verb|a_slice = append(a_slice, 40)|, we now have that both the capacity of the underlying array and its length increased to 1. If we instead created a slice such as \verb|var a_slice = []int{1, 2, 3, 4}|, then we would have a capacity and a length of both \verb|4|.
\nl
Go also provides a more granular way to control the creation of slices, and that's through the \verb|make()| function. This function has the following syntax:

\begin{codedefine}
    \begin{lstlisting}[language = go]
make(
    []type,
    length,
    capacity
)\end{lstlisting}
\tcblower
Parameters of \verb|make()|:
\begin{itemize}
    \item \verb|[]type|: the type of the slice;
    \item \verb|length|: the length of the slice;
    \item \verb|capacity|: the capacity of the slice.
\end{itemize}
\end{codedefine}

\pagebreak
For instance, with \verb|var a_slice = make(int[], 0, 7)| we are creating a slice of capacity 7 and with a length of 0. Note that slices can assume a \verb|nil| value (which is equivalent to C's \verb|NULL| or Python's \verb|None|) if they are declared but never assigned, but if they get declared through \verb|make()| they will be equal to a slice of \verb|nil|s. A clearer example follows:

\begin{codeblock}{SlicesAndNil.go}
    \begin{lstlisting}[language = go]
// This slice is equal to nil
var nilSlice []int

// This slice is not equal to nil, but contains nil
var sliceOfNils = make([]int, 0)

// This last slice will have the same length of nilSlice, but different content\end{lstlisting}
\end{codeblock}

We can also have the equivalent of Python's dictionary in Go, and they are called \textbf{maps}. A map can either be declared but not assigned (and this must be done with the \verb|make()| function only) or declared and assigned. Why can we only declare with \verb|make()|? Let's try to run the following code:

\begin{codeblock}{MapsError.go}
    \begin{lstlisting}[language = go]
package main

func main() {
    var aMap map[string]int

    aMap["Hello"] = 2
}
    \end{lstlisting}
\end{codeblock}

If we try to execute this program, we will obtain the following error:

\begin{terminal}
    \begin{lstlisting}[style = notexterm]
 $ ./MapsError
    \end{lstlisting}
    \begin{tcolorbox}
        \begin{lstlisting}[basewidth=0.44em, numbers=none]
panic: assignment to entry in nil map

goroutine 1 [running]:
main.main()
        /home/user/go/MapsError.go:8 +0x28\end{lstlisting}
    \end{tcolorbox}
\end{terminal}

This is because the \verb|var aMap map[string]int| instruction just declared the map, but never allocated it. So, in order to allocate it, we must use \verb|make()|. We can then do the following:
\pagebreak
\begin{codeblock}{MapsExamples.go}
    \begin{lstlisting}[language = go]
package main

func main() {
    var aMap = map[string]int {
        "Mark": 27,
        "Bob": 21,
        "Silvia": 19,
        "Mike": 26
    }
}\end{lstlisting}
\end{codeblock}